\section{Actions}
Just like HTML is generated from Elements, Actions are rendered into
JavaScript to handle events raised in the browser. Actions always
transformed into JavaScript and sent through WebSockets pipe.

\subsection*{Direct Wiring}
There are two type of actions. First class is direct JavaScript
strings given directly as Erlang lists or via JavaScript/OTP
transformations.

\vspace{1\baselineskip}
\begin{lstlisting}
    wf:wire("window.location='http://synrc.com'").
\end{lstlisting}

\subsection*{Actions Render}
Second class is actions which are in fact Erlang records being
rendered during page load, server events or client events.

\vspace{1\baselineskip}
\begin{lstlisting}
    wf:wire(#redirect{url="http://synrc.com"}).
\end{lstlisting}
\vspace{1\baselineskip}

This section describes actions that are presented as Erlang records,
they are called Nitrogen actions and this set is quite reduced in compare to Nitrogen.
You don't need to remember a lot of Erlang records, ussually you just
use JavaScript/OTP.

\paragraph{}
However basic N2O actions which are a part of N2O API, just like {\bf wf:redirect},
are implemented as Erlang records as given in example. In case you need deffered
render of JavaScript you can use Erlang records instead of direct wiring with
Erlang lists or JavaScript/OTP.

\paragraph{}
Any action wired with {\bf wf:wire} is enveloped with {\bf \#wire\{actions=[]\}}
which is also action capable to polymorph render of custom or built-in actions specified in list.
Following action embedding are also valid:

\vspace{1\baselineskip}
\begin{lstlisting}
    wf:wire(#wire{actions=[#redirect{url="http://synrc.com"}]}).
\end{lstlisting}
\vspace{1\baselineskip}

\subsection{#update}
Update action is main action for manipulating DOM. You have several functions
which is useful for positioning insertion inside DOM element:

\vspace{1\baselineskip}
\begin{lstlisting}
    wf:update(Target, Elements)
    wf:insert_top(Target, Elements)
    wf:insert_bottom(Target, Elements)
    wf:insert_before(Target, Elements)
    wf:insert_after(Target, Elements)
    wf:remove(Target)
\end{lstlisting}
\vspace{1\baselineskip}

\subsection{Page Event}

\subsection{Control Event}

\subsection{API Event}

\subsection{Function}
