\section{Architecture}

\subsection*{Reduced Latency}
The secret of reduced latency is simple. We try to deliver rendered HTML
as soon as possible and render JavaScript only after WebSocket initialization.
We use three steps and three erlang processes for achieve that.

\includeimage{page-lifetime.png}{Page Lifetime}

\paragraph{}
During lifetime N2O uses HTTP process for serving first HTML page after that it dies
and spawns Transition process. After that browser initiates WebSocket connections
to the same URL end-point and N2O creates persistent WebSocket process and
Transition Process dies.

\paragraph{}
Also your page could spawn processes with {\bf wf:comet}. They are also persistent
processes that acts as regullar Erlang processes. It is common way to organize
non-blocking UI for things like file uploads and long-term operations.

\subsection{HTTP Process}
In the first HTTP handler we only render HTML and all created javascript actions are stored in transition process.

\vspace{1\baselineskip}
\begin{lstlisting}
        transition(Actions) ->
            receive {'N2O',Pid} -> Pid ! Actions end.
\end{lstlisting}

\paragraph{}
HTTP handler dies immediately after returning HTML. Transition process
waits for request from WebSocket handler.

\subsection{Transition Process}
Just after receiving HTML browser initiates WebSocket connection
and this starts WebSocket handler on the server. After returning
javascript actions transition process dies and only WebSocket handler stays.
At this point initialization phase is done.

\subsection{WebSocket Process}
After that all client/server communication is done via WebSocket channel.
All events that come from the browser are handeled by N2O, which renders elements
to HTML and actions to JavaScript. Each user at a time have only one WebSocket process per connection.

\subsection{Async Processes}
These are user processes which are created by {\bf wf:comet}. This is the legacy name
taken from times where async technology was called COMET for XHR channel. Async processes
are optional and needed only when you have UI event which takes a lot of time to proceed
like gigabyte file uploads, etc. You can create many Async Processes per user.

