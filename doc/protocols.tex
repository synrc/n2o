\section{Protocols}
N2O is more that just web framework or even application server.
It also has protocol specification that covers broad range of application domains.
In this chapter we go deep inside network capabilities of N2O communications.
N2O protocol also has an ASN.1 formal description, however here we will speak on it freely.
Here is the landscape of N2O protocols stack.

\includeimage{n2o-proto.png}{Protocols Stack}

\paragraph{}
You may find it similar to XML-based XMPP, binary COM/CORBA,
JSON-based WAMP, Apache Camel or Microsoft WCF communication foundations.
We took best from all and put into one protocols stack for web,
social and enterprise domains providing stable and mature implementation for Erlang
in a form of N2O application server.

\newpage
\subsection*{Cross Language Compatibility}
N2O application server implemented to support N2O protocol protocol defenition
in Erlang which is widely used in enterprise applications.
Experimental implementation in Haskell {\bf n2o.hs} exists
which supports only core {\bf heart} protocol along with {\bf bert} formatter.
We will show you how N2O clients are compatible across
different server implementations in different languages.

\subsection*{Web Protocols: {\bf nitro}, {\bf spa}, {\bf bin}}
N2O protocols stack provides definition for several unoverlapped protocol layers.
N2O application server implementaion of N2O protocol specification supports
four protocol layers from this stack for WebSocket and IoT applications: {\bf heart}, {\bf nitro}, {\bf spa} and {\bf bin} protocols.
HEART protocol is designed for reliable managed connections and stream channel initialization.
The domain of NITRO protocol is HTML5 client/server interoperability, HTML events and JavaScript delivery.
SPA protocol dedicated for games and static page applications that involves no HTML,
such as SVG based games or non-gui IoT applications.
And finally binary file transfer protocol for images and gigabyte file uploads and downloads.
All these protocols transfers coexist in the same multi-channel stream.

\subsection*{Social Protocols: {\bf roster}, {\bf muc}, {\bf search}}
For social connectivity one may need to use {\bf synrc/roster} instant messaging server
that supports {\bf roster} protocol  with variation
for enabling public rooms {\bf muc} or full-text {\bf search} facilities.

\subsection*{Enterprise Protocols: {\bf bpe}, {\bf mq}, {\bf rest}}
There is no single system shipped to support all of N2O protocols but it
could exist theoretically. For other protocols implementation you may refer
to other products like {\bf spawnproc/bpe}, {\bf synrc/rest} or {\bf synrc/mq}.

\newpage
\subsection*{Channel Termination Formatters}
N2O protocol is formatter agnostic and it doesn't strict you
to use a particular encoder/decoder.
Application developers could choose their own formatter per protocol.

\vspace{1\baselineskip}
\begin{lstlisting}
    1. BERT : {io,"fire();",1}
    2. WAMP : [io,"fire();",1]
    3. JSON : {name:io,eval:"fire();",data:1}
    4. TEXT : IO \xFF fire(); \xFF 1\n
    5. XML  : <io><eval>fire();</eval><data>1</data></io>
\end{lstlisting}
\vspace{1\baselineskip}

E.g. N2O uses TEXT formatting for ``PING'' and ``N2O,'' protocol messages,
across versions N2O used to have IO message formatted with JSON and BERT both.
All other protocol messages were BERT from origin.
Make sure formatters set for client and server is compatible.

\vspace{1\baselineskip}
\begin{lstlisting}
    > application:set_env(n2o,formatter,bert).
\end{lstlisting}
\vspace{1\baselineskip}

Note that you may include to support more that one protocol on the client.
At server side you can change formatter on the fly without breaking
the channel stream. Each message during data stream could be formatted
using only one protocol at a time. If you want to pass each message
through more that one formatter you should write an echo protocol.

\vspace{1\baselineskip}
\begin{lstlisting}
    <script src='/n2o/protocols/bert.js'></script>
    <script src='/n2o/protocols/client.js'></script>
    <script>protos = [ $bert, $client ]; N2O_start();</script>
\end{lstlisting}


\newpage
\subsection*{Protocol Loop}
After message arrives to endpoint and handlers chain is being initializes,
message then comes to protocol stack. N2O selects appropropriative protocol
module and handle the message. After than message is being formatted and
replied back to stream channel. Note that protocol loop is applicable
only to WebSocket stream channel endpoint.

\includeimage{n2o_protocols.png}{Messaging Pipeline}

\paragraph{}
Here is pseudocode how message travels for each protocol until some
of them handle the message. Note tnat this logic is subject to change.

\vspace{1\baselineskip}
\begin{lstlisting}[caption=Top-level protocol loop in {n2o}\_{proto}]
    reply(M,R,S)              -> {reply,M,R,S}.
    nop(R,S)                  -> {reply,<<>>,R,S}.
    push(_,R,S,[],_Acc)       -> nop(R,S);
    push(M,R,S,[H|T],Acc)     ->
        case H:info(M,R,S) of
              {unknown,_,_,_} -> push(M,R,S,T,Acc);
             {reply,M1,R1,S1} -> reply(M1,R1,S1);
                            A -> push(M,R,S,T,[A|Acc]) end.
\end{lstlisting}
\vspace{1\baselineskip}

\newpage
\subsection*{Enabling Protocols}
You may set up protocol from sys.config file,
enabling or disabling some of them on the fly.

\vspace{1\baselineskip}
\begin{lstlisting}
    protocols() ->
       wf:config(n2o,protocols,[ n2o_heart,
                                 n2o_nitrogen,
                                 n2o_client,
                                 n2o_file  ]).

\end{lstlisting}
\vspace{1\baselineskip}

For example in Skyline (DSL) application you use only {\bf nitro} protocol:

\vspace{1\baselineskip}
\begin{lstlisting}
    > wf:config(n2o,protocols).
    [n2o_heart,n2o_nitrogen]
\end{lstlisting}
\vspace{1\baselineskip}

And in Games (SPA) application you need only {\bf spa} protocol:

\vspace{1\baselineskip}
\begin{lstlisting}
    > wf:config(n2o,protocols).
    [n2o_heart,n2o_client]
\end{lstlisting}
\vspace{1\baselineskip}

\newpage
\subsection{Heartbeat}

N2O also provides basic Heartbeat protocol that can be formatted at your whim.
Heartbeat protocol is essential WebSocket application level protocol for
PING and N2O initialization. It pings every 4-5 seconds from client-side to server
thus allowing to determine client online presence. On reconnection or initial connect
client sends N2O init marker telling to server to reinitialize the context.

\vspace{1\baselineskip}
\begin{lstlisting}
    ws.send('PING');
    ws.send('N2O,');
\end{lstlisting}
\vspace{1\baselineskip}

You can try manually send these messages in web console to see whats happening,
also you can enable logging the heartbeat protocol by including its module in log\_modules:

\vspace{1\baselineskip}
\begin{lstlisting}
    log_modules() -> [n2o_heart].
\end{lstlisting}
\vspace{1\baselineskip}

Heartbeat protocol PING request returns empty message NOP binary response;
N2O messages returns JSON with EVAL fields of rendered actions.

\newpage
\subsection{Nitrogen Compatibility Layer}

NITRO protocol consist of three protocol messages: {\bf pickle}, {\bf flush} and {\bf direct}.
Pickled messages are used if you send messages over unencrypted
channel and want to hide the content of the message,
that was generated on server. You can use BASE64 pickling mechanisms
with optional AES/RIPEMD160 encrypting.

\vspace{1\baselineskip}
\begin{lstlisting}
    ws.send(enc(tuple(atom('pickle'),
        binary('ddtake'),
        binary('g2gCaAVkAAJldmQABWluZGV4ZAAEdGFrZWsABH'+
                Rha2VkAAVldmVudGgDYgAABXViAAQKXmIAC3cK'),
        [tuple(atom('ddtake'),'0')])));
\end{lstlisting}
\vspace{1\baselineskip}

Where Base64 represents the N2O EVENT:

\vspace{1\baselineskip}
\begin{lstlisting}
    #ev{module=index,msg=take,trigger="take",name=event}
\end{lstlisting}
\vspace{1\baselineskip}

This is Nitrogen-based messaging model. Nitrogen WebSocket processes receive also
flush and delivery protocol messages, but originated from server. These are internal Nitrogen
protocol messages. This request will return JSON with EVAL field only.

\newpage
\subsection{Client/Server}

Client messages usually originated at client and represent the Client API Requests:

\vspace{1\baselineskip}
\begin{lstlisting}
    ws.send(enc(tuple(
        atom('client'),
        tuple(atom('join_game'),1000001))));
\end{lstlisting}
\vspace{1\baselineskip}

Server messages are usually being sent to client originated on the
server by sending {\bf info} notifications directly to Web Socket process:

\vspace{1\baselineskip}
\begin{lstlisting}
    > WebSocketPid ! {server, Message}
\end{lstlisting}
\vspace{1\baselineskip}

You can obtain this Pid like here:

\vspace{1\baselineskip}
\begin{lstlisting}
    event(init) -> wf:info(?MODULE, "Web Socket Pid: ~p",[self()]);
\end{lstlisting}
\vspace{1\baselineskip}

You can also send server messages from client relays and vice versa.
But it is up to your application and client/server handlers how to handle these messages.

\vspace{1\baselineskip}
\begin{lstlisting}
    ws.send(enc(tuple(
        atom('server'),
        tuple(atom('attach'),1000001))));
\end{lstlisting}
\vspace{1\baselineskip}

NOTE: client/server request may return JSON with EVAL and DATA fields.

\newpage
\subsection*{JSON enveloped EVAL and DATA}

Each message from Web Socket channel to Client is encoded as JSON object.
\footahref{https://github.com/5HT/n2o/blob/master/priv/n2o.js}{N2O.js}
is used to decode WebSocket binary messages from JSON container.

\begin{lstlisting}
    { "eval": "ws.send("Send Back This String");",
      "data": [ 131,104,2,100,0,7,109,101,115,115,
                97,103,101,107,0,5,72,101,108,108,111 ] }
\end{lstlisting}

EVAL values are evaluated immediately and DATA values are passed
to handle\_web\_socket(data) function if it exists.

\begin{lstlisting}
    function handle_web_socket(body) { console.log(body); }
\end{lstlisting}

\subsection*{JSON enveloped BERT}

Usually in DATA come BERT messages (Binary Erlang Term Format).
\footahref{https://github.com/5HT/n2o/blob/master/priv/bert.js}{BERT.js}
is used to decode application protocol message.

\begin{lstlisting}
    function handle_web_socket(body) {
        console.log(String(dec(body))); }
\end{lstlisting}

\begin{lstlisting}
    E> Received: {message,"Hello"}
\end{lstlisting}

\subsection{Binary}

When you need raw binary Blob on client-side,
for images or other raw data, you can ask server like this:

\vspace{1\baselineskip}
\begin{lstlisting}
    ws.send(enc(tuple(
        atom('bin'),
        binary('API Request'));
\end{lstlisting}
\vspace{1\baselineskip}

And handle also in binary clause:

\vspace{1\baselineskip}
\begin{lstlisting}
    event({binary,Message}) ->
        wf:info(?MODULE, "This API will return Raw Binary", []),
        <<84,0,0,0,108>>;
\end{lstlisting}
\vspace{1\baselineskip}

NOTE: if event returns not the binary, client will receive BERT encoded message.

\subsection*{BERT}

Erlang RPC protocol interconnection with JavaScript nodes should be transferred as BERT answers.

\begin{lstlisting}
    function handle_web_socket(body) {
        console.log(String(dec(body))); }
\end{lstlisting}

\subsection*{RAW Binary}

Raw images for fastest possible speed should be transferred as binary answers.

\begin{lstlisting}
    function handle_web_socket_blob(body) { }
\end{lstlisting}

\begin{lstlisting}
    E> Unknown Raw Binary Received: [72,101,108,108,111]
\end{lstlisting}
